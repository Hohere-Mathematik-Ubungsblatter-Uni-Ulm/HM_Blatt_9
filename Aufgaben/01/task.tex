\section{Aufgabe 1}
    \subsection{a)} 
         Wir untersuchen $\sum_{n=2}^{\infty}\frac{1}{\sqrt[3]{n^2-1}}=\sum_{n=2}^{\infty}a_n$ auf Konvergenz. \\
         D.h. wir untersuchen zuerst $a_n$ für $n \to \infty$:
         $$n^2-1 \sim n^2 \Rightarrow \sqrt[3]{n^2-1} \sim \sqrt[3]{n^2}, \text{ das bedeutet, dass } a_n=\frac{1}{\sqrt[3]{n^2-1}} \sim \frac{1}{\sqrt[3]{n^2}}=\frac{1}{(n)^{\frac{2}{3}}}$$
         Wir wissen, dass Reihen der Form $\sum_{k=2}^{\infty}\frac{1}{k^x}$ für $x>1$ konvergieren. \\
         $$\frac{2}{3}<1 \Rightarrow \sum_{n=2}^{\infty}\frac{1}{\sqrt[3]{n^2}} \text{ divergiert.}$$
         Da $a_n \sim \frac{1}{n^{\frac{2}{3}}} \Rightarrow \sum_{n=2}^{\infty}\frac{1}{\sqrt[3]{n^2-1}}$ divergiert.

    \subsection{b)}
        Wir untersuchen $\sum_{n=0}^{\infty} (-1)^n \frac{\sqrt{n}}{n+1} = \sum_{n=0}^{\infty}(-1)^nb_n$ auf Konvergenz. \\
        Wir überprüfen mit dem Leibniz-Kriterium:
        \subsubsection{Monoton fallend}
        Wir betrachten die Ableitung der Funktion $b_n = \frac{\sqrt{n}}{n+1}$ \\
        $$
        b_n' = \frac{(n+1) \cdot \frac{1}{2\sqrt{n}} - \sqrt{n} \cdot 1}{(n+1)^2} = \frac{(n+1) \cdot \frac{1}{2\sqrt{n}} - \sqrt{n}}{(n+1)^2} = \frac{\frac{n+1}{2\sqrt{n}} - \frac{2\sqrt{n}}{2}}{(n+1)^2} = \frac{\frac{-n + 1}{2\sqrt{n}}}{(n+1)^2}
        $$
        Für große \( n \) ist der Zähler negativ (\( -n + 1 < 0 \)), daher ist \( b_n' < 0 \). Das bedeutet, dass \( b_n \) monoton fallend ist.
        
        \subsubsection{Grenzwert von \( b_n \)}
        Wir berechnen den Grenzwert von \( b_n \) für \( n \to \infty \):
        \[
        \lim_{n \to \infty} b_n = \lim_{n \to \infty} \frac{\sqrt{n}}{n+1}=  \lim_{n \to \infty} \frac{\sqrt{n}}{n+1} = \lim_{n \to \infty} \frac{1}{\sqrt{n} + \frac{1}{\sqrt{n}}}
        \]
        Da \( \sqrt{n} \to \infty \) für \( n \to \infty \), folgt:
        \[
        \lim_{n \to \infty} \frac{1}{\sqrt{n} + \frac{1}{\sqrt{n}}} = 0
        \]
        Da alle Bedingungen für das Leibniz-Kriterium erfüllt sind, konvergiert die Reihe $\sum_{n=0}^{\infty} (-1)^n \frac{\sqrt{n}}{n+1}$        

    \subsection{c)}
        Wir untersuchen $\sum_{n=0}^{\infty} \left( \frac{n}{n+1} \right)^{n^2}$ auf Konvergenz: \\
        Der Summand ist gegeben durch $a_n = \left( \frac{n}{n+1} \right)^{n^2}$.
        
        Wir betrachten zunächst den Bruch \( \frac{n}{n+1} \). Für große \( n \) gilt:$\frac{n}{n+1} = 1 - \frac{1}{n+1}$ \\
        Der natürliche Logarithmus wird für kleine x approximiert mit:$\ln(1-x) \sim -x \quad \text{für } x \to 0$ \\
        \[
        \Rightarrow \ln\left( \frac{n}{n+1} \right) = \ln\left( 1 - \frac{1}{n+1} \right) \sim -\frac{1}{n+1}.
        \]
        Der Logarithmus des Summanden \( a_n \) ist somit $\ln(a_n) = n^2 \cdot \ln\left( \frac{n}{n+1} \right)$ \\
        Mit der Approximation \( \ln\left( \frac{n}{n+1} \right) \sim -\frac{1}{n+1} \) ergibt sich somit $\ln(a_n) \sim n^2 \cdot \left( -\frac{1}{n+1} \right)$ \\
        Für große \( n \) ist \( n+1 \sim n \), damit:
        \[
        \ln(a_n) \sim -\frac{n^2}{n} = -n \Rightarrow a_n \sim e^{-n}
        \]
        Da sich für große \( n \) der Summand \( a_n \) wie \( e^{-n} \) verhält, können wir die Reihe approximieren mit
        \[
        \sum_{n=0}^{\infty} e^{-n}.
        \]
        Diese Reihe ist eine geometrische Reihe der Form $\sum_{n=0}^{\infty} r^n$ mit $r = e^{-1}$
        Da \( e^{-1} \) eine konstante Zahl kleiner als \( 1 \) ist (\( e^{-1} \approx 0.367 \)), konvergiert die geometrische Reihe und damit auch die gegebene Reihe.        
        
    \subsection{d)}
        Wir untersuchen $\sum_{n=0}^{\infty} \frac{2^n + n}{3^n}$ auf Konvergenz. \\
        Der Summand lässt sich wie folgt aufteilen:
        \[
        \frac{2^n + n}{3^n} = \frac{2^n}{3^n} + \frac{n}{3^n}\Rightarrow \sum_{n=0}^{\infty} \frac{2^n + n}{3^n} = \sum_{n=0}^{\infty} \frac{2^n}{3^n} + \sum_{n=0}^{\infty} \frac{n}{3^n}
        \]
        Die erste Reihe lautet $\sum_{n=0}^{\infty} \frac{2^n}{3^n}$. Dies ist eine geometrische Reihe der Form $\sum_{n=0}^{\infty} r^n$ mit $r = \frac{2}{3}$ \\
        Da \( |r| = \frac{2}{3} < 1 \), konvergiert die geometrische Reihe. \\
        Der Wert der geometrischen Reihe ist gegeben durch:
        \[
        \sum_{n=0}^{\infty} r^n = \frac{1}{1-r} \Rightarrow \sum_{n=0}^{\infty} \left( \frac{2}{3} \right)^n = \frac{1}{1 - \frac{2}{3}} = \frac{1}{\frac{1}{3}} = 3
        \]
        Die zweite Reihe lautet $\sum_{n=0}^{\infty} \frac{n}{3^n}$
        Wir überprüfen die Konvergenz mithilfe des Quotientenkriteriums. Der Summand lautet $a_n = \frac{n}{3^n}$ und der nächste Summand ist $a_{n+1} = \frac{n+1}{3^{n+1}}$ \\
        Wir setzen \( a_n \) und \( a_{n+1} \) in den Quotientenkriterium ein:
        \[
        \frac{a_{n+1}}{a_n} = \frac{\frac{n+1}{3^{n+1}}}{\frac{n}{3^n}} = \frac{n+1}{3n} = \frac{1+\frac{1}{n}}{3} \Rightarrow \lim_{n \to \infty} \frac{1+\frac{1}{n}}{3} = \frac{1}{3}
        \]
        Da \( L = \frac{1}{3} < 1 \), konvergiert die Reihe \( \sum \frac{n}{3^n} \) nach dem Quotientenkriterium. \\
        
        $\Rightarrow$ Da beide Teilreihen \( \sum_{n=0}^{\infty} \frac{2^n}{3^n} \) und \( \sum_{n=0}^{\infty} \frac{n}{3^n} \) konvergieren, konvergiert auch die gegebene Reihe.
        
        
    \subsection{e)}
        Wir untersuchen $ \sum_{n=1}^{\infty} \frac{n!}{n^n}$ auf Konvergenz. Hierbei nutzen wir ein weiteres Mal das Quotientenkriterium. \\
        $$a_n = \frac{n!}{n^n} \Rightarrow a_{n+1} = \frac{(n+1)!}{(n+1)^{n+1}}$$
        $$\Rightarrow L = \lim_{n \to \infty}\left|\frac{a_{n+1}}{a_n}\right|
        =\lim_{n \to \infty}\left| \frac{\frac{(n+1)!}{(n+1)^{n+1}}}{\frac{n!}{n^n}} \right| 
        = \lim_{n \to \infty}\left| \frac{n^n}{(n+1)^n} \cdot \frac{n+1}{n+1} \right|
        = \lim_{n \to \infty}\left( \frac{n}{n+1} \right)^n$$
        Wir wissen, dass $\lim_{n \to \infty}\left(1-\frac{1}{n}\right)^n=\frac{1}{e} \Rightarrow L = \frac{1}{e}$ \\
        Da $L = \frac{1}{e} < 1$ ist, konvergiert die Reihe absolut.

    \subsection{f)}
        Wir untersuchen$\sum_{n=0}^{\infty} \frac{1}{\binom{4n}{3n}}$ auf Konvergenz.
        $$ \sum_{n=0}^{\infty} \frac{1}{\binom{4n}{3n}} = \sum_{n=0}^{\infty} \frac{1}{\frac{(4n)!}{(3n)!n!}}=\sum_{n=0}^{\infty} \frac{(3n)!n!}{(4n)!}$$
        Um das Konvergenzverhalten zu bestimmen, analysieren wir das asymptotische Verhalten von \(a_n\) für große \(n\). \\
        Wir wissen: $k! \sim \sqrt{2\pi k} \left( \frac{k}{e} \right)^k$ für großes $k$. \\
        Wir wenden diese Näherung auf \( (4n)! \), \( (3n)! \) und \( n! \) an:
        \[
        (4n)! \sim \sqrt{8\pi n} \left( \frac{4n}{e} \right)^{4n}, \quad (3n)! \sim \sqrt{6\pi n} \left( \frac{3n}{e} \right)^{3n}, \quad n! \sim \sqrt{2\pi n} \left( \frac{n}{e} \right)^n.
        \]
        \[
        \Rightarrow a_n \sim \frac{\sqrt{6\pi n} \, \left( \frac{3n}{e} \right)^{3n} \cdot \sqrt{2\pi n} \, \left( \frac{n}{e} \right)^n}{\sqrt{8\pi n} \, \left( \frac{4n}{e} \right)^{4n}}.
        \]

        Wir vereinfachen den Bruch zu:
        \[
        \frac{\sqrt{6\pi n} \cdot \sqrt{2\pi n}}{\sqrt{8\pi n}} = \sqrt{\frac{3\pi n}{2}}.
        \]
        und
        \[
        \frac{\left( \frac{3n}{e} \right)^{3n} \cdot \left( \frac{n}{e} \right)^n}{\left( \frac{4n}{e} \right)^{4n}} = \frac{(3n)^{3n} \cdot n^n}{(4n)^{4n}}.
        \]
        sowie
        \[
        \frac{(3n)^{3n} \cdot n^n}{(4n)^{4n}} = \frac{3^{3n} \, n^{3n} \, n^n}{4^{4n} \, n^{4n}} = \frac{3^{3n}}{4^{4n}}.
        \]

        \[
        \Rightarrow a_n \sim \sqrt{\frac{3\pi n}{2}} \cdot \left( \frac{3}{4} \right)^{4n}.
        \]
        
        Wir verwenden nun ein weiteres Mal das Quotientenkriterium: \\                
        Da der dominante Faktor in \(a_n\) der Term \(\left( \frac{3}{4} \right)^{4n}\) ist, ergibt sich:
        \[
        \frac{a_{n+1}}{a_n} \sim \left( \frac{3}{4} \right)^4.
        \]
        
        Da \( \frac{3}{4} < 1 \), folgt \( \left( \frac{3}{4} \right)^4 < 1 \). Somit konvergiert die Reihe nach dem Quotientenkriterium.
    